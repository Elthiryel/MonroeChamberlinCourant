\chapter{Implemented Algorithms}
\label{cha:implementedAlgorithms}

In this chapter we present implemented algorithms for the utilitarian versions of Monroe and Chamberlin-Courant multiwinner voting rules in the satisfaction-based framework.
\\

\noindent
\textbf{Proposition 1 (Implicit in the paper of Betzler et al. \cite{3}).} Let $\alpha$ be a normal DFSF, $N$ be a set of agents, $A$ be a set of alternatives, $V$ be a preference profile of $N$ over $A$, and $S$ a $K$-element subset of $A$ (where $K$ divides $\norm{N}$). Then there is a polynomial-time-algorithm that computes a (possibly partial) optimal K-assignment $\Phi^{S}_{\alpha}$ (Monroe K-assignment $\Phi^{S}_{\alpha}$) of the agents to the alternatives from $S$.
\\

\section{Algorithm A}

Algorithm A was first presented by Skowron et al. \cite{1} and tries to solve $\alpha$-Monroe-SatWinner. It builds a solution iteratively (greedily). In each step we pick some not-yet-assigned alternative $a_{i}$ (using some criterion) and assign it to those $\frac{N}{K}$ agents that are not assigned to any other alternative yet and whose satisfaction of being matched with $a_{i}$ is maximal. This algorithm runs in polynomial time. Pseudocode is presented in Algorithm 1.

\begin{algorithm}
\caption{Algorithm A}\label{euclid}
\begin{algorithmic}[1]
	\Procedure{ComputeMonroeSatWinner}{}
		\State $\Phi \gets$ a map defining a partial assignment, iteratively built by the algorithm
		\State $\Phi^{\leftarrow} \gets$ the set of agents for which the assignment is already defined
		\State $\Phi^{\rightarrow} \gets$ the set of alternatives already used in the assignment
		\If {$K \leq 2$}
			\State compute the optimal solution using an algorithm of Betzler et al. \cite{1} and return
		\EndIf
		\State $\Phi$ = $\{\}$
		\For{$i \gets 1$ to $K$}
			\State $score \gets \{\}$
			\State $bests \gets \{\}$
			\ForAll{$a_{i} \in A \setminus \Phi^{\rightarrow}$}
				\State $agents \gets$ sort $N \setminus \Phi^{\leftarrow}$ so that if agent $j$ preceeds agent $j'$ then $pos_{j}(a_{i}) \leq pos_{j'}(a_{i})$
				\State $bests[a_{i}] \gets$ choose first $\frac{N}{K}$ elements from $agents$
				\State $score[a_{i}] \gets \sum_{j \in bests[a_{i}]}(m - pos_{j}(a_{i}))$
			\EndFor
			\State $a_{best} \gets argmax_{a \in A \setminus \Phi^{\rightarrow}} score[a]$
			\ForAll{$j \in bests[a_{best}]$}
				\State $\Phi[j] \gets a_{best}$
			\EndFor
		\EndFor
	\EndProcedure
\end{algorithmic}
\end{algorithm}

\section{Algorithm B}

Algorithm B is an extension to Algorithm A and was presented in the same paper. The idea is to run Algorithm A first and then, using Proposition 1, optimally reassign the alternatives to the voters. It should noticeably improve the results of the algorithm and it still runs in polynomial time.

\section{Algorithm C}

Algorithm C is a further improvement over Algorithm B, also presented by Skowron et al. \cite{1}. The idea is that instead of keeping only one partial function $\Phi$ that is iteratively extended up to the full assignment, we keep a list of up to $d$ partial assignment functions, where $d$ is a parameter of the algorithm. At each iteration, for each assignment function $\Phi$ among the $d$ stored ones and for each alternative $a$ that does not yet have agents assigned to by this $\Phi$, we compute an optimal extension of this $\Phi$ that assigns to $a$. As a result we obtain possibly more than $d$ (partial) assignment functions. For the next iteration we keep those $d$ of them that give highest satisfaction. If we take $d = 1$, we obtain Algorithm B. Pseudocode is presented in Algorithm 3.
\\

Unlike previous algorithms, Algorithm C can be used for both Monroe and Chamberlin-Courant rules. To adapt it for the Chamberlin-Courant rule, we have to replace the contents of the first for all loop with the appropriate code, presented in Algorithm 2.

\begin{algorithm}
\caption{Algorithm C - for all code replacement}\label{euclid}
\begin{algorithmic}[1]
	\ForAll{$a_{i} \in A \setminus \Phi^{\rightarrow}$}
		\State $\Phi' \gets \Phi$
		\ForAll{$j \in N$}
			\If{agent $j$ prefers $a_{i}$ to $\Phi'(j)$}
				\State $\Phi'(j) \gets a_{i}$
			\EndIf
		\EndFor
		\State $newPar.push(\Phi')$
	\EndFor
\end{algorithmic}
\end{algorithm}

\begin{algorithm}
\caption{Algorithm C}\label{euclid}
\begin{algorithmic}[1]
	\Procedure{ComputeMonroeSatWinner}{}
		\State $\Phi \gets$ a map defining a partial assignment, iteratively built by the algorithm
		\State $\Phi^{\leftarrow} \gets$ the set of agents for which the assignment is already defined
		\State $\Phi^{\rightarrow} \gets$ the set of alternatives already used in the assignment
		\State $Par \gets$ a list of partial representation functions
		\State $Par = []$
		\State $Par.push(/{/})$
		\For{$i \gets 1$ to $K$}
			\State $newPar = []$
			\For{$\Phi \in Par$}
				\State $bests \gets \{\}$
				\ForAll{$a_{i} \in A \setminus \Phi^{\rightarrow}$}
					\State $agents \gets$ sort $N \setminus \Phi^{\leftarrow}$ (agent $j$ preceeds agent $j'$ implies that $pos_{j}(a_{i}) \leq pos_{j'}(a_{i})$
					\State $bests[a_{i}] \gets$ choose first $\frac{N}{K}$ elements of $agents$
					\State $\Phi' \gets \Phi$
					\ForAll{$j \in bests[a_{i}]$}
						\State $\Phi'[j] \gets a_{i}$
					\EndFor
					\State $newPar.push(\Phi')$
				\EndFor
			\EndFor
			\State sort $newPar$ according to descending order of the total satisfaction of the assigned agents
			\State $Par \gets$ choose first $d$ elements of $newPar$
		\EndFor
		\For{$\Phi \in Par$}
			\State $\Phi \gets$ compute the optimal representative function using an algorithm of Betzler et al. \cite{3} for the set of winners $\Phi^{\rightarrow}$
		\EndFor
		\State \Return the best representative function from $Par$
	\EndProcedure
\end{algorithmic}
\end{algorithm}

\section{Algorithm R}

As shown by Skowron et al. \cite{1}, algorithms A, B and C achieve very high approximations ratios under linear satisfaction function for the cases where $K$ is small relative to $m$. For the remaining cases, we can use a sampling-based randomized algorithm (called Algorithm R). We expect that under nonlinear satisfaction function algorithms should behave analogously in relation to each other.
\\

The idea of this algorithm is to randomly pick $K$ alternatives and match them optimally to the agents, using Proposition 1. Such an algorithm may be very unlucky and pick $K$ alternatives that all of the agents rank low. Yet, if $K$ is comparable to $m$ then it is likely that such a random sample would include a large chunk of some optimal solution. Algorithm can naturally be used for both Monroe and Chamberlin-Courant systems.

\section{Algorithm AR}

Algorithm family A-C and algorithm R are naturally suitable for different cases. Therefore, Skowron et al. \cite{1} proposed to combine algorithms A and R. Pseudocode is presented in Algorithm 4.
\\

TODO: requires further description

\begin{algorithm}
\caption{Algorithm AR}\label{euclid}
\begin{algorithmic}[1]
	\Procedure{ComputeMonroeSatWinner}{}
		\State $\lambda \gets$ required probability of achieving the approximation ratio equal $0.715 - e$
		\If{$\frac{H_{K}}{K} \geq \frac{e}{2}$}
			\State compute the optimal solution using an algorithm of Betzler et al. \cite{3} and return
		\EndIf
		\If{$m \leq 1 + \frac{2}{e}$}
			\State compute the optimal solution using a simple brute force algorithm and return
		\EndIf
		\State $\Phi_{1} \gets$ solution returned by Algorithm A
		\State $\Phi_{2} \gets$ run the sampling-based algorithm - $\log (1 - \lambda) \cdot \frac{2 + e}{e}$ times; select the assignment of the best quality
		\State \Return the better assignment among $\Phi_{1}$ and $\Phi_{2}$
	\EndProcedure
\end{algorithmic}
\end{algorithm}

\section{Algorithm GM}

Algorithm GM (greedy marginal improvement) was introduced by Lu and Boutilier \cite{4} for the Chamberlin-Courant rule. It was generalized by Skowron et al. \cite{1} to apply it to the Monroe rule as well, for which it can be viewed as an extension to Algorithm B.
\\

We start with an empty set $S$. Then we execute $K$ iterations. In each iteration we find an alternative $a$ that is not assigned to agents yet, and that maximizes the value $\Phi^{S \cup \{a\}}_{\alpha}$. It requires a large number of computations of $\Phi^{S}_{\alpha}$, which is a notable disadvantage for the Monroe case, as a computation is a slow process based on min-cost/max-flow algorithm \cite{3}. Pseudocode is presented in Algorithm 5.

\begin{algorithm}
\caption{Algorithm GM}\label{euclid}
\begin{algorithmic}[1]
	\Procedure{ComputeSatWinner}{}
		\State $\Phi^{S}_{\alpha}$ - the partial assignment that assigns a single alternative to at most $\frac{n}{K}$ agents, that assigns to the agents only the alternatives from $S$, and that maximizes the utilitarian satisfaction $l^{\alpha}_{sum}(\Phi^{S}_{\alpha})$
		\State $S \gets \emptyset$
		\For{$i \gets 1$ to $K$}
			\State $a \gets argmax_{a \in A \setminus S} l^{\alpha}_{sum} (\Phi^{S \cup \{\alpha\}}_{\alpha})$
			\State $S \gets S \cup \{a\}$
		\EndFor
		\State \Return $\Phi^{S}_{\alpha}$
	\EndProcedure
\end{algorithmic}
\end{algorithm}

\section{Algorithm P}

This algorithm, introduced by Skowron et al. \cite{1}, computes a certain value $x$ and greedily computes an assignment that (approximately) maximizes the number of agents assigned to one of their top-$x$ alternatives. If after this process some agent has no alternative assigned, we assign her to the most preferred alternative from those already picked.
\\

$w(x)$ used in the algorithm is a Lambert's W-function, defined to be the solution of the equation $x = w(x)e^{w(x)}$. Pseudocode of Algorithm P is presented in Algorithm 6. It applies to Chamberlin-Courant rule.

\begin{algorithm}
\caption{Algorithm P}\label{euclid}
\begin{algorithmic}[1]
	\Procedure{ComputeCCSatWinner}{}
		\State $\Phi \gets$ a map defining a partial assignment, iteratively built by the algorithm
		\State $\Phi^{\leftarrow} \gets$ the set of agents for which the assignment is already defined
		\State $\Phi^{\rightarrow} \gets$ the set of alternatives already used in the assignment
		\State $num\_pos_{x}(a) \gets \norm{\left\{ i \in [n] \setminus \Phi^{\leftarrow} : pos_{i}(a) \leq x \right\}}$ - the number of not-yet assigned agents that rank alternative $a$ in one of their first $x$ positions
		\State $w(\cdot)$ - Lambert's W-function
		\State $\Phi = \{\}$
		\State $x = \left\lceil \frac{mw(K)}{K} \right\rceil$
		\For{$i \gets 1$ to K}
			\State $a_{i} \gets argmax_{a \in A \setminus \Phi^{\rightarrow}} num\_pos_{x}(a)$
			\ForAll{$j \in [n] \setminus \Phi^{\leftarrow}$}
				\If{$pos_{j}(a_{i}) < x$}
					\State $\Phi[j] \gets a_{i}$
				\EndIf
			\EndFor
		\EndFor
		\ForAll{$j \in A \setminus \Phi^{\leftarrow}$}
			\State $a \gets$ such server from $\Phi^{\rightarrow}$ that $\forall_{a' \in \Phi^{\rightarrow}} pos_{j}(a) \leq pos_{j}(a')$
			\State $\Phi[j] \gets a$
		\EndFor
	\EndProcedure
\end{algorithmic}
\end{algorithm}

\section{Genetic Algorithm}

This algorithm starts with an initial set of creatures (each of them presenting a possible solution under Chamberlin-Courant rule) which are later mutated and crossed over with each other. The best creatures (ones with the highest total satisfaction) are preferred for further mutation and crossover in order to better investigate neighbourhood of local maxima, but on the other hand algorithm also produces new creatures by crossing over random existing ones to better explore entire solution space, not limiting to local extrema.
\\

In each iteration of the algorithm creatures are evaluated (satisfaction is computed). Best creature in terms of total satisfaction is compared with currently best found creature and takes its place if it is better. Half of the evaluated creatures (the best ones) are chosen for further propagation. Each of them is then mutated randomly. Remaining creatures are created by crossing over random creatures from the "better" half with each other. Resulting set of creatures is used for the next iteration. Pseudocode of the algorithm is presented in Algorithm 7.

\begin{algorithm}
\caption{Genetic Algorithm}\label{euclid}
\begin{algorithmic}[1]
	\Procedure{ComputeCCSatWinner}{}
		\State $I$ - number of iterations
		\State $c$ - number of creatures
		\State $\Phi_{best}$ - best creature (preference profile)
		\State $creatures \gets$ generate initial random set of $c$ creatures
		\For{$i \gets 1$ to I}
			\State $creaturesSorted \gets$ sort $creatures$ by total satisfaction
			\If{$satisfaction(creaturesSorted[1]) > satisfaction(\Phi_{best})$}
				\State $\Phi_{best} = creaturesSorted[1]$
			\EndIf
			\State $bestCreatures \gets$ choose first $c/2$ elements from $creaturesSorted$
			\State $mutated \gets$ mutate all creatures from $bestCreatures$ randomly
			\State $crossed \gets$ crossover random creatures from $bestCreatures$ to produce $c/2$-element set
			\State $newCreatures \gets mutated \cup crossed$
			\State $creatures \gets newCreatures$
		\EndFor
	\EndProcedure
\end{algorithmic}
\end{algorithm}

\section{Simulated Annealing}

TODO
