\chapter{Introduction}
\label{cha:introduction}

We study the effectiveness of algorithms for approximate winner determination under the Monroe and Chamberlin-Courant multiwinner voting rules using nonlinear satisfaction function. Both of the rules aim to select a group of candidates that best represent the voters. Having good voting rules and algorithms for them is important, as multiwinner elections are used both in human societies (e.g. parliament elections) and software systems (e.g. recommendation systems). Rules studied in this paper are particularly interesting because they have both desired features of multiwinner rules: they provide accountability (there is connection between the elected candidates and the voters, so each voter has a representative assigned to her and each candidate knows who she represents) and proportional representation of the voters’ views.

We assume that candidates participate in the election with multiple winners (a commitee with multiple members is selected) and they are elected by voters, each of whom ranks all the candidates (each voter provides a linear order over the set of candidates expressing their preferences). For each voter the Monroe and Chamberlin-Courant rules assign a single candidate as their representative (with some constraints, which are detailed in the further part of the thesis).

The candidates are selected and assigned to the voters optimally, either by maximizing the total satisfaction of all voters, or by minimizing the total dissatisfaction of all voters.
The total satisfaction is calculated as a sum of individual satisfactions of the voters. We assume that there is a  satisfaction function that measures how well a voter is represented by the candidate. The function is the same for each voter. It is a decreasing function, so a voter is more satisfied if the candidate assigned to her is ranked higher. The dissatisfaction is calculated in the similar way, except that the function is an increasing one. In this paper I study cases in which the satisfaction function is a nonlinear one.

The main drawback of the aforementioned rules is that election winner determination is NP-hard under each of them \cite{2} which makes them hard to use in practice, as it would force the use of algorithms that don’t provide an optimal result for every data set. Therefore, using these systems for real-life elections may rise some difficulties. However, they can be used for the recommendation systems conveniently, as a good but not optimal recommendation is still useful. Skowron et al. \cite{1} provided approximation algorithms for both the rules that return near-optimal results for various test cases (including real-life data and synthetic data), but for linear satisfaction function only.

In this paper we focus on providing several algorithms for the Monroe and Chamberlin-Courant rules using nonlinear satisfaction function and evaluating them empirically against various data sets. For smaller data, results can be easily assessed by comparing them to the optimal result (calculated with the brute-force algorithm). For bigger data, the upper bound of the optimal result must be used for comparison. We implement and evaluate various heuristic algorithms, as well as the existing approximation algorithms for the linear satisfaction function, but applying them to the nonlinear cases.

