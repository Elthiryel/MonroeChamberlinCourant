\chapter{Introduction}
\label{cha:introduction}

We study the effectiveness of algorithms for approximate winner determination under the \linebreak Monroe \cite{8} and Chamberlin-Courant \cite{9} multiwinner voting rules using nonlinear satisfaction function. The purpose of both these rules is to select a group of candidates that best represent the voters. Having good voting rules and algorithms for them is important, because multiwinner elections are used both in human societies (e.g., for parliament elections) and particular software systems (e.g., in recommendation systems \cite{4}). Rules studied in this paper are exceptionally interesting because they have two desired features of multiwinner rules: they provide accountability (there is a direct connection between the elected candidates and the voters, so each voter has a representative assigned to her and each candidate knows who she represents) and proportional representation of the voters’ views.

We assume that candidates participate in the election with multiple winners (a committee with multiple members is selected) and they are elected by voters, each of whom ranks all the candidates (each voter provides a linear order over the set of candidates expressing her preferences). For each voter the Monroe and Chamberlin-Courant rules assign a single candidate as her representative (with some constraints, which are detailed in the further part of the thesis).

The candidates are selected and assigned to the voters optimally, by maximizing the total satisfaction of all the voters. The total satisfaction is calculated as a sum of individual satisfactions of the voters. We assume that there is a  satisfaction function that measures how well a voter is represented by the candidate. The function is the same for each voter. It is a decreasing function, so a voter is more satisfied if the candidate assigned to her is ranked higher. In this thesis we study cases in which the satisfaction function is a nonlinear one.

Monroe and Chamberlin-Courant systems may be potentially very useful, because they are one of the few multiwinner election systems that provide both accountability and proportional results. Most of the currently used voting rules lack at least one of these properties. For example, D'Hondt method used to elect members of Polish lower house of parliament lacks accountability (specific parliament members are accountable to the political parties, not to the voters), while single-member constituency plurality system used for United Kingdom parliament elections lacks proportionality.

The main drawback of the aforementioned rules is that their winner determination problems are NP-hard \cite{3, 4, 2} which makes them hard to use in practice, as it would force the use of algorithms that do not provide an optimal result for every data set. Therefore, using these systems for real-life elections may rise some difficulties. However, they can be used for the recommendation systems conveniently, as a good but not optimal recommendation is still useful. Lu and Boutilier \cite{4} and Skowron et al. \cite{1} provided approximation algorithms for both the rules that return near-optimal results for various test cases (including real-life data and synthetic data), but for linear satisfaction functions only.

In this thesis we focus on providing several algorithms for the Monroe and Chamberlin-Courant rules using nonlinear satisfaction functions and evaluating them empirically against various data sets. For smaller data, results can be easily assessed by comparing them to the optimal result (calculated with the brute-force algorithm). For bigger data, the upper bound of the optimal result must be used for comparison. As finding optimal solution is NP-hard \cite{3, 4, 2}, there is a need to provide good algorithms which can compute results that are suboptimal, but still as close to optimal as possible. We implement and evaluate various heuristic algorithms, as well as the existing approximation algorithms for the linear satisfaction function, but applying them to the nonlinear cases.

Skowron et. al \cite{1} have already provided approximation algorithms for these systems, but only under linear satisfaction function. In this thesis we focus on providing algorithms for non-linear satisfaction functions, as they can better reflect real preferences of the voters.

\section{Results}

TODO
