\chapter{Preliminaries}
\label{cha:preliminaries}

In the first part of the chapter we explain basic notions regarding multiwinner election problem. Next, we present definitions of winner determination problem under the Monroe and Chamberlin-Courant voting rules. Finally, we present some notions regarding approximate solutions to the problem.

\section{Basic notions}

\noindent
\textbf{Preferences.} \hspace{.1in} For each $n \in \mathbb{N}$, by $[n]$ we mean $\left\{ 1, \ldots, n \right\}$. We assume that there is a set $N = [n]$ of \textit{agents} and a set $A = \left\{ a_{1}, \ldots a_{m} \right\}$ of \textit{alternatives}. Each agent $i$ has a strict linear order $\succ_{i}$ over $A$, called \textit{preference order} ($a_{\pi(1)} \succ_{i} a_{\pi(2)} \succ_{i} \ldots \succ_{i} a_{\pi(m)}$ for some permutation $\pi$ of $[m]$). For an alternative $a$, $pos_{i}(a)$ represents the position of $a$ in the preference order of agent $i$. For example, if $a$ is the most preferred alternative for $i$ then $pos_{i}(a) = 1$, and if $a$ is the least preferred one then $pos_{i}(a) = m$. A collection $V = (\succ_{1}, \ldots, \succ_{n})$ of agents' preference orders is called a \textit{preference profile}.
\\

If $A$ is the set of all the alternatives and $B$ is a nonempty strict subset of $A$, then by $B \succ A - B$ we mean that all alternatives in $B$ are preferred to those outside of $B$ for the preference order $\succ$.
\\

\noindent
\textbf{Positional scoring function.} \hspace{.1in} A \textit{positional scoring function} (PSF) is a function $\alpha^{m}: [m] \rightarrow \mathbb{N}$. It assigns a score to every position in the agent's preference order. If for each $i,j \in [m]$, if $i < j$ then $\alpha^{m}(i) < \alpha^{m}(j)$, a PSF $\alpha^{m}$ is an \textit{increasing positional scoring function} (IPSF). Analogously, if for each $i,j \in [m]$, if $i < j$ then $\alpha^{m}(i) > \alpha^{m}(j)$, a PSF $\alpha^{m}$ is a \textit{decreasing positional scoring function} (DPFS) .
\\

If $\alpha^{m}$ is an IPSF then $\alpha^{m}(i)$ can represent the \textit{dissatisfaction} value of an agent if she is assigned to an alternative ranked $i$'th in her preference order. We assume that for each IPSF $\alpha^{m}(1) = 0$, so an agent is not dissatisfied with her best alternative. Analogously, if $\alpha^{m}$ is a DPSF, $\alpha^{m}$ can represent an agent's \textit{satisfaction} value. We assume that for each DPSF $\alpha^{m}(m) = 0$, so an agent is not satisfied at all with her worst alternative. In some cases, we will write $\alpha$ instead of $\alpha^{m}$ to simplify notation.
\\

An IPSF (DPSF) $\alpha^{m}$ is considered a \textit{linear dissatisfaction (satisfaction) function} if for each $i \in [m-1]$ value $\alpha^{m}(i+1) - \alpha^{m}(i)$ is constant. Every IPSF (DPSF) not satisfying that constraint is considered a \textit{nonlinear dissatisfaction (satisfaction) function}.
\\

We define a family $\alpha$ of IPSFs (DPSFs) of the form $\alpha = (\alpha^{m})_{m \in \mathbb{N}}$, where $\alpha^{m}$ is an IPSF (a DPSF) on $[m]$, such that:

\begin{enumerate}
	\item For a family of IPSFs it holds that $\alpha^{m+1}(i) = \alpha^{m}(i)$ for all $m \in \mathbb{N}$ and $i \in [m]$.
	\item For a family of DPSFs it holds that $\alpha^{m+1}(i+1) = \alpha^{m}(i)$ for all $m \in \mathbb{N}$ and $i \in [m]$.
\end{enumerate}

We build families of IPSFs (DPSFs) iteratively by appending (prepending) values to functions with smaller domains (smaller $m$), leaving values from the previous function unchanged. To simplify notation, we will refer to such families of IPSFs (DPSFs) as \textit{normal} IPSFs (DPSFs).
\\

\noindent
\textbf{Assignment functions.} \hspace{.1in} We define a \textit{K-assignment function} as any function $\Phi: N \rightarrow A$ that satisfies $\norm{\Phi(N)} \leq K$ (it assigns agents to no more than $K$ alternatives). A \textit{Monroe K-assignment function} is a K-assignment function that additionally satisfies the following constraint: For each alternative $a \in A$ we have that either $\left\lfloor \frac{\norm{N}}{K} \right\rfloor \leq \norm{\Phi^{-1}(a)} \leq \left\lceil \frac{\norm{N}}{K} \right\rceil$ or $\norm{\Phi^{-1}(a)} = 0$.
\\

A \textit{partial K-assignment function} is similar to a K-assignment function, with one exception: it may assign a null alternative, $\bot$, to some of the agents. A \textit{partial Monroe K-assignment} is a partial K-assignment that can be extended to a regular Monroe K-assignment by replacing each assignment of a null alternative with one from $A$. If we have an assignment function $\Phi$, alternative $\Phi(i)$ is called the \textit{representative} of agent $i$.
\\

Let $S$ be a set of alternatives. By $\Phi^{S}$ we mean a (possibly partial) K-assignment function that assigns agents only to alternatives from $S$.
\\

We assume that $\alpha$ is a normal IPSF (DPSF). Each of the following functions assings a positive integer to a given assignment $\Phi$:
\begin{gather}
	l^{\alpha}_{sum}(\Phi) = \sum^{n}_{i=1} \alpha (pos_{i}(\Phi(i))),\\
	l^{\alpha}_{max}(\Phi) = max^{n}_{i=1} \alpha (pos_{i}(\Phi(i))),\\
	l^{\alpha}_{min}(\Phi) = min^{n}_{i=1} \alpha (pos_{i}(\Phi(i))).
\end{gather}
These functions combine dissatisfaction (satisfaction) of the agents to assess the quality of the assignment for the entire society. In the utilitarian framework (which is used in this thesis), the first function (which simply calculates the sum of individual agents dissatisfaction or satisfaction value) is used as the \textit{total dissatisfaction function} in the IPSF case and as the \textit{total satisfaction function} in the DPSF case. We use the second and the third functions, respectively, as the total dissatisfaction and satisfaction functions for IPSF and DPSF cases in the egalitarian framework, where the most dissatisfied (the least satisfied) agent is decisive.
\\

For each subset of the alternatives $S \subseteq A$ such that $\norm{S} \leq K$, $\Phi^{S}_{\alpha}$ denotes the partial (Monroe) K-assignment that assigns agents only to the alternatives from $S$ and such that $\Phi^{S}_{\alpha}$ maximizes the satisfaction $l^{\alpha}_{sum}(\Phi^{S}_{\alpha})$.

\section{Monroe and Chamberlin-Courant rules}

We will now define the problems of winner determination under the Monroe and Chamberlin-Courant (abbreviated as CC) rules. The goal is to find an optimal assignment function, where by the optimal function we accept one that either maximizes the total satisfaction or minimizes the total dissatisfaction.
\\

\noindent
\textbf{DisWinner.} Let $\alpha$ be a normal IPSF. An instance of $\alpha$\textit{-CC-DisWinner} problem is defined as follows: given a set of agents $N = [n]$, a set of alternatives $A = \left\{ a_{1}, \ldots, a_{m} \right\}$, a preference profile $V$ of the agents, and positive integer $K$ representing commitee size, we seek a K-assignment function $\Phi$ that minimizes $l^{\alpha}_{sum}(\Phi)$. The problem $\alpha$\textit{-Monroe-DisWinner} is defined in the same way but we additionally require $\Phi$ to be a Monroe K-assignment function.
\\

\noindent
\textbf{SatWinner.} Let $\alpha$ be a normal DPSF. The problem $\alpha$\textit{-CC-SatWinner} is defined in the same way as $\alpha$\textit{-CC-DisWinner}, except that we seek a K-assignment $\Phi$ that maximizes $l^{\alpha}_{sum}(\Phi)$. The problem $\alpha$\textit{-Monroe-SatWinner} is defined in the same way but we additionally require $\Phi$ to be a Monroe K-assignment function.
\\

In terms of optimal solutions, dissatisfaction-based problems and satisfaction-based ones are equivalent, as a DPSF can always be transformed to an IPSF and vice versa.
\\

These problems' intention is to find a (Monroe) K-assignment function which returns a set of $K$ alternatives, who are viewed as the winners of the given multiwinner election (e.g. elected members of a commitee).

\section{Approximate solutions}

As for many normal IPSFs multiwinner election problems under both Monroe and Chamberlin-Courant rules are NP-hard \cite{2,3}, we seek approximate solutions.
\\

\noindent
\textbf{Approximation algorithms.} Let $r$ be a real number such that $r \geq 1 (0 < r \leq 1)$, let $\alpha$ be a normal IPSF (DPSF), and let $R$ be either Monroe or CC. An algorithm is an r-approximation algorithm for $\alpha$-R-DisWinner problem ($\alpha$-R-SatWinner problem) if on each instance $I$ it returns a feasible assignment $\Phi$ such that $l^{\alpha}_{sum}(\Phi) \leq r \cdot OPT$ (such that $l^{\alpha}_{sum}(\Phi) \geq r \cdot OPT$), where $OPT$ is the optimal total dissatisfaction (satisfaction) $l^{\alpha}_{sum}(\Phi_{OPT})$.
\\
